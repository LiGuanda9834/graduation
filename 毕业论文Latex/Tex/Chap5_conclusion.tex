
\chapter{总结与展望}
本文主要工作内容围绕武器目标分配问题展开,首先针对经典的武器目标分配问题进行了线性化,并对线性化后的模型设计了一个基于列生成框架并用外逼近方法求解列生成子问题的算法,再将问题和算法进行拓展到复杂资源约束和动态的武器目标分配问题,并给出了动态武器目标分配问题的一种评价方式。

本文在第二节中,介绍了武器目标分配问题的经典模型及一些已经存在的精确算法,同时利用了经典模型的凸性,给出了两种不同的凸优化建模方式,并应用外逼近求解框架对问题进行求解。除此以外,还给出了一种问题的线性化建模方式,将问题等价地转化为一个线性化的模型,后续章节中的基于列生成的求解算法都是在这个线性化模型的基础上展开的。最后结合实际的作战需求,给出了问题在多作战要素下的推广,将武器目标分配问题拓展为武器-雷达-目标分配问题,并给出了问题的数学模型构建与线性化方法。

在第三章中,详细地介绍了武器-雷达-目标分配问题的数学模型与线性化方式,并给出了一个例子来详细地展示线性化模型的含义。在线性化模型的基础上,针对问题具有指数多列的特点,设计了基于列生成框架的求解算法。在这个具体的问题中,列生成的子问题并非一个简单的线性规划,而是可以被转化为一个凸优化问题,它的形式与武器目标分配问题相似,但是却实现了目标之间的分离,在缩减了问题规模的同时可以对问题进行并行化。在针对子问题的算法设计上,我们提出了两种不同的建模方式,并使用外逼近的求解框架来对问题进行求解。为了能够增加问题的求解速度,我们针对子问题设计了基于数学特性和实际含义的有效不等式,并且对经典的列生成方法进行了改进,允许子问题选择非最优的子问题并且可以同时选择多个不同的改进列。最后应用数值实验证明了这个框架和求解技巧的效果。

本文在第四章中,介绍了如何将静态武器目标分配拓展到动态武器目标分配问题,给出了问题的场景描述并提出了对应的数学模型。相比于经典的动态武器目标分配问题,这个问题采用了非均匀的时间离散,希望在不过多牺牲问题内容信息的前提下,建立一个能够精确求解的模型。基于建立的模型,我们将之前提到的列生成求解框架进行推广,此时子问题依然是一个凸优化问题。针对子问题,我们证明了采用对数方式进行建模的模型会更紧,并基于此完成了算法的设计。

在武器目标分配问题上,本文还有很多可能改进的地方。首先是这个问题与分支定界框架更有效的结合,在使用传统的分支定界方法来进行求解过程中,在分支过程中会增加对于某一个变量的限制,但是在这个问题中它会导致子问题的求解失去凸性,从而影响问题的求解。其次在子问题的求解上,还有一些可以选择的加速技巧没有使用,同时同一目标的多个子问题之间非线性项相同,不同之处仅仅在于线性项系数,如何有效利用这一点来实现热启动以实现子问题的求解加速也是一个有趣的研究课题。最后在列生成过程中使用的非最优子问题选择方式也依然有改进空间,如何自适应地选择需要的列依然有改进的空间。