\chapter{引言}\label{chap:introduction}


\section{武器目标分配问题}
\subsection{静态武器目标分配问题}
武器目标分配问题是一个军事领域中经典的数学优化问题,最初由Manne(1958)[1] 引入,最开始的目标为为来袭导弹分配可用的拦截弹,以最小化导弹摧毁受保护物资的概率。随着对问题研究的深入与算力的进步,武器目标分配逐渐发展出很多不同的细分领域,在众多的划分标准中,一个重要的划分标准为考虑一阶段问题还是多阶段问题,即将问题分为了静态武器目标分配问题(静态武器目标分配问题)与动态武器目标分配问题(动态武器目标分配问题)。静态武器目标分配问题中不考虑时间,仅考虑针对一次导弹来袭所做的拦截。而动态武器目标分配问题则主要考虑拦截 - 观察 - 拦截的模式,即每次打击过后观察打击的结果并安排下一次打击。Lloyd和Witsenhausen(1986)[2] 证明了即使是较为简单的静态武器目标分配仍然是NP完全的,因此在之前的研究中,大多数研究者的研究课题是寻找合适的智能方法,通过启发式算法在一个较短的时间内寻找接近最优解的方法。但启发式算法并不能在理论上保证最终结果的最优性,在之前的研究中,使用整数规划方法求解问题精确解的研究内容较少,求解规模较小,直到近几年才有相对有效的算法出现。

在比较早的时期,由于计算机解决大型非线性问题的能力有限 (Day, 1966)[3],当时的研究者将研究种地放在了形式较为简单的静态武器目标分配问题 (denBroeder et al., 1959)[4],即:

\begin{alignat}{3}
    &\min\quad && \sum_{j=1}^n V_j \left( \prod_{i=1}^m (1 - p_{ij}) ^ {x_{ij}}  \right) \tag{P1.1} \\ 
    &\text{s. t.}\quad &&\sum_{j=1}^n x_{ij} \leq l_i, \quad &\hspace{1cm}\forall ~ i \in I, \tag{P1.2}\\
    & && x_{ij} \in \mathbb{N}, \quad &\hspace{1cm}\forall ~ j\in J, ~ i \in I. \tag{P1.3}
\end{alignat}
其中约束表示每个武器的使用上限不能超过其容量。
他们结合当时的计算能力开发了二相对应的比较简单的求解方法(Lemus and David,1963)[5],(Day,1966)[3]。 Eckler and Burr (1972)[6] 提出并讨论了采用动态规划的方式求解 静态武器目标分配 的可能性,但未能较好地得到解决此类问题的算法。随着计算能力的提高,计算机解决更加复杂的问题的能力也在提高。Burr et al. (1985)[7] 建模并解决了最早的 动态武器目标分配 问题之一,与此同时Chang (1987)[8] , Soland (1987)[9] 和 Hosein(1989)[10] 等人也对求解动态武器目标分配问题提供了一些方法。同时,用新方法解决了具有较少假设的 静态武器目标分配 模型(包括Kwon et al., (1999)[11];Metler et al., (1990)[12];Wacholder, (1989)[13]),较少的假设使得问题与现实的情况更加贴合,但是同样增加了问题的难度。这种对于早期算法进行修补的过程一直持续到 2000年。在20世纪00年代,模型和求解算法得到了进一步发展:一个发展方向是在模型中引入更多参数,使得模型能够更好地反应现实情况(即 Shang et al. (2007)[14] 和Karasakal (2008)[15]);另一个发展方向则是保证模型能够更加快速地得到最优解或者最优解的近似解(即 Malcolm, (2004)[16],Ahuja et al. (2007)[17],以及 Ahner and Parson (2015)[18])。随着新的模型的建立,后续研究者基于这些模型开发了更新的算法(即 Bertsekas et al., (2000)[19];Kline et al., (2017)[20];Wu et al., (2008)[21] )或将原有的方法进行了效果显著的改进(即 Ahuja et al., (2007)[17];Lee et al., (2002)[22];Su et al., (2008)[23];Xin et al., (2010)[24])。

\subsection{动态武器目标分配问题与多作战元素的武器目标分配问题}
随着计算机算力的不断增长,武器目标分配现有的研究方向除了在已经存在的现有模型上改进求解方法之外,还提出了关于目标飞行路径的时间依赖性的动态模型 (Khosla, (2001)[25]; Leboucher et al., (2013)[26]),但与现有模型相比受到的关注较少,且这些模型中解决方案技术的改进尚未出现。

第一个模型是 shoot-look-shoot 模型,它对于每次打击结果都可以进行观察,因此在打击的过程中目标是否中存在都是确定的。在这种模型中,假设问题共有两阶段,武器在第一阶段中将武器分配给目标,之后随着打击结束对问题进行一次观察。针对第一次打击带来的结果,将剩余的武器分配给幸存的目标。

第二个模型被称为两阶段模型,或者更一般地说多阶段模型,它与 shoot-look-shoot 的不同之处在于它不允许在打击后观察,且每个目标仅被考虑一次,若在某一阶段没被摧毁,后续将不再有摧毁该目标的机会。也就是说,在 shoot-look-shoot 问题中,作战目的是对给定数量的目标进行反复攻击,直到所有目标都被摧毁或达到迭代次数的限制。 而在2阶段问题中,第一次会出现的目标是给定的,但是在第二阶段,有可能会重新出现新的目标,且目标的数量和类型仅仅能够给出一个概率分布而不是确定性的。在已知多阶段的目标的概率分布的前提下,希望能够给出对目标毁伤的数学期望最高的方案。
可以看出,动态武器目标分配问题的复杂程度明显高于静态武器目标分配,但是却能够更好地结合实际情况。在更加复杂的模型中,也存在将两种动态模型结合在一起的情况,但是目前针对此问题尚没有比较好的精确求解方法。


在动态武器目标分配问题发展的过程中,关于问题的困难程度也曾经被多人指出。Khosla (2001)[25] 指出,“尽管采用了两步法,但每个优化问题仍然即使对于数量不多的威胁、武器系统和时间点,也有巨大的搜索空间。”改进两步法的方法尚未出现。同样,Leboucher at el. (2013) [26]评论了问题的指数增长并提出了一种两步解决技术,并补充说另一个问题是“如何评估一个解的质量”。武器目标分配 的未来将需要解决前面提到的以调度为中心的 动态武器目标分配 的困难与能够利用问题的特殊结构的技术。此外,存在许多问题中本应该被引入的参数,由于它们会带来增加的计算复杂性而被删除,随着计算能力的增加,这些参数可以使用新的理论模型重新引入,以保证模型更加贴合实际情况。

除了从时间的维度将问题拓展为动态武器目标分配问题,具体的作战场景刻画也有。在现代化的作战场景中,除了武器与目标两个关键要素外,多种战场资源的综合调配将起到关键性的作用。举例而言,在防空反导作战当中,一次基本的打击包含雷达对目标的锁定,武器对目标的打击和指挥所对整体作战场景的控制,因此将这些作战资源进行抽象,针对每个来袭目标,我们可以为其设计一条或多条“目标 - 雷达 - 指控 - 发射车 - 弹药”的杀伤链,其中,指控部分可以解耦,而弹药与发射车则是绑定在一起的,相比于传统的武器目标分配问题多了雷达这个作战要素。针对这个更复杂的场景,对应的数学模型也需要进行对应的调整,其数学形式为:


\optimalProblem{\max}{\sumFromTo{k=1}{n}{v_k\zkh{1-\prod_{i=1}^{m}{\prod_{j=1}^{t}{\xkh{1-p_{ijk}}^{x_{ijk}}}}}}}{\sumFromTo{j=1}{t}{\sumFromTo{k=1}{n}{x_{ijk}}\leq l_i,\quad \forall i = 1,\cdots,m\\ & \sumFromTo{i=1}{m}{\sumFromTo{k=1}{n}{x_{ijk}}\leq r_j},\quad \forall j = 1,\cdots\,t}\\& x_{ijk} \in \N}
其中两个约束分别代表武器和雷达不能超过其容量限制。

因此,在武器目标分配这个研究课题上,无论是进一步优化对于静态武器目标分配问题的求解方法以增加求解规模,还是将求解方法拓展到动态武器目标分配问题与更复杂作战场景下,为这些问题设计高效、使用的求解算法都具有研究意义。



\section{预备知识}
\subsection{整数规划问题与分支定界求解框架}
整数规划问题是数学规划的一个重要分支,它在工程设计、生产计划、资源分配、物流管理等领域有着广泛的应用。整数规划问题的特点是决策变量必须取整数值,这使得问题的求解相比于连续规划更为复杂。

根据决策变量的不同,整数规划问题可以分为纯整数规划问题、混合整数规划问题和0-1整数规划问题。纯整数规划问题要求所有决策变量均为整数,而混合整数规划问题则允许部分变量为连续变量。0-1整数规划问题是纯整数规划的一个特例,其中所有决策变量只能取0或1,常用于表示选择与否的决策情形。从约束和目标函数的形式来看,整数规划问题又可以分为整数线性规划问题(ILP)和非线性整数规划问题(INLP)。整数线性规划问题的目标函数和约束条件都是线性的,这是最常见的整数规划类型。非线性整数规划问题则涉及非线性的目标函数或约束条件,凸整数规划问题作为一种特殊的非线性整数规划问题,它的目标函数是凸函数且约束条件构成的可行域为凸集。

由于整数解的空间是离散的,整数规划问题的求解通常不能直接应用连续优化的方法。常用的求解方法包括分支定界法\cite{land1960automatic}、割平面法(\cite{gomory1958outline}, \cite{crowder1983solving})和启发式算法等。在现代的求解软件中,通常是在分支定界的框架下结合多种割平面与启发式方法来对问题进行求解。

\begin{definition}
与整数规划相关的基本定义
\begin{itemize}
    \item \textbf{分数变量 (Fractional Variable)}: 在线性松弛问题的解中,如果一个整数变量的值不是整数,则称该变量为分数变量。
    
    \item \textbf{上界 (Upper Bound, UB)}: 对于最小化问题,上界是目标函数的一个估计值,该值大于或等于最优解的目标函数值,通常由一个可行解计算得到。
    
    \item \textbf{下界 (Lower Bound, LB)}: 对于最小化问题,下界是目标函数的一个估计值,该值小于或等于最优解的目标函数值,通常由一个松弛解得到。
    
    \item \textbf{相对间隙 (Relative Gap)}: 相对间隙用于衡量上界和下界之间的差距相对于上界的比例,定义为
    \[
    \text{Relative Gap} = \frac{\text{UB} - \text{LB}}{|\text{UB}|} \times 100\%
    \]
    其中,$\text{UB}$是上界,$\text{LB}$是下界。
    \item \textbf{刻面 (Facet)}: 对于一个多面体(如线性规划的可行域),一个刻面是多面体的极大面,即不存在其他的多面体的一部分。由于刻面能够提供多面体的最紧凑表示,它对于描述多面体的结构非常重要。

    \item \textbf{有效不等式 (Valid Inequality)}: 对于一个给定的整数规划问题,一个有效不等式是指一个不等式,它对于该问题的所有可行整数解都成立。有效不等式用于增强线性松弛的表达,从而更紧密地逼近整数可行域,提高求解效率。
\end{itemize}
\end{definition}

分支定界方法被证明是一个精确算法,它是一种对可行解进行穷举的算法,但是和普通穷举法所不同的是,分支定界算法在对某一分支进行检索之前会预先计算出该分支的上界或下界,如果界限无法比目前的最优可行解更好,那么该分支就会被舍弃,通过降低搜索空间的规模节约了大量的搜索时间。而割平面则可以通过割掉不可行的部分,来让解空间逼近整数可行集的凸包,因此二者能够相互结合。在本文中会用到分支定界的部分知识,但是仅仅在外逼近算法中有与经典分支定界方法有所不同的敌方,因此在此处并不详述分支定界方法的算法流程。



\subsection{列生成方法}
\textbf{列生成算法概述}

列生成方法是一种在大规模优化问题中广泛应用的技术,它的核心思想是在一个较小的、更易于管理的问题(即主问题)上求解,而不是在问题的所有可能性上一次性求解。这种方法主要应用在大规模的线性规划和整数规划问题中,尤其适用于那些具有大量可能决策变量,但在任何最优解中只有少数决策变量非零的情况。

\textbf{列生成算法的算法流程}

列生成方法求解的是一个标准的线性规划问题:
\begin{align*}
    \min \quad & c^T x \\
    \text{s.t.} \quad & Ax = b, \\
                       & x \geq 0,
\end{align*}
其中 $x$ 代表的决策变量个数非常多,通过以下步骤迭代地求解这个线性规划问题:

\textbf{步骤一:初始化主问题。}

这一步通常通过启发式方法或随机选择的方式,选取一个较小的、可解的问题作为主问题的初始版本,主问题的数学模型可以描述为:
\begin{align*}
    \min \quad & c_B^T x_B \\
    \text{s.t.} \quad & A_B x_B = b, \\
                       & x_B \geq 0,
\end{align*}
使用 $x_B$ 表示仅选择了原问题的部分列。

\textbf{步骤二:求解主问题。}

接下来我们使用线性规划方法(如单纯形法或者内点法)来求解这个初步限制的主问题,获取当前的最优解和对偶变量值。

\textbf{步骤三:求解子问题。}

对于子问题,我们实际上是在尝试寻找那些未被包含在当前主问题中的列(即尚未选取的变量),若加入主问题后有可能改进当前解。这一过程被称为定价过程。在定价过程中,我们会计算每个未选取变量的约化费用(Reduced cost):
\begin{equation*}
    \min \quad c_i - \pi^T A_i,
\end{equation*}
若该问题的最优值小于0,对应的 $x$ 将被加入到主问题的备选解中。

\textbf{步骤四:判断当前主问题是否达到最优解。}

如果在步骤三的定价过程中,我们找到了一个约化费用小于零的决策变量,那么我们就将这个决策变量(也即新的列)加入到主问题中,并回到步骤二,对扩充了的主问题进行求解。否则,如果所有未选取的决策变量的约化费用都大于或等于零,那么我们就停止算法,因为这表明当前的主问题解就是全局最优解。

\begin{algorithm}[!htbp]
    \small
    \caption{列生成算法}\label{alg:column_generation}
    \begin{algorithmic}[1]
        \Procedure{列生成}{$A, b, c$}\Comment{输入:线性规划参数 $A, b, c$}
        \State 初始化主问题的变量集合 $S \subset \{1,2,\ldots,n\}$
        \Repeat
            \State 求解主问题 $\min\{c^T x : Ax = b, x \geq 0, x_i = 0 \text{ 对于所有 } i \notin S\}$
            \State 得到当前解 $x^*$ 和对偶变量值 $\pi^*$
            \State 求解子问题 $\min\{c_j - \pi^* A_j : j \notin S\}$
            \If{$\exists j \notin S$ 使得 $c_j - \pi^* A_j < 0$}
                \State $S = S \Cup \dkh{j}$
            \EndIf
        \Until{不存在 $j \notin S$ 使得 $c_j - \pi^* A_j < 0$}
        \State \textbf{return} 主问题的最优解 $x^*$\Comment{输出:线性规划的最优解}
        \EndProcedure
    \end{algorithmic}
\end{algorithm}

列生成方法的最优性证明及它在武器目标分配问题中的具体表现形式将在后续章节中进行陈述。

\subsection{外逼近方法}
\textbf{外逼近算法概述}
外逼近方法是一种求解凸优化问题的技术,如果目标函数或约束条件满足凸性条件的同时形式又比较复杂,则希望通过逐步构造可行域的一个线性外逼近来近似原问题的可行域,从而将问题转化为迭代地求解一系列线性规划。当这个方法应用在凸整数规划问题(属于MINLP的一种特殊情况)上时,根据已有的结论,仅仅考虑那些整数点的外逼近割就足够完成对于原问题的等价转化,将这个方法与分支定界框架进行有效的集合,则可以在有限的步骤内得到问题的精确解。

\textbf{原问题与等价转化}

考虑一个凸优化问题:
\begin{align*}
    \min \quad & f(x) \\
    \text{s.t.} \quad & g_i(x) \leq 0, \quad i = 1, \ldots, m, \\
                       & x \in X,
\end{align*}
其中 $f(x)$ 是凸函数,$g_i(x)$ 是凸约束,$X$ 是凸集。为了将原问题转化为一个具有线性目标函数的优化问题,我们引入一个新变量 $\eta$ 并考虑以下问题:
\begin{align*}
    \min \quad & \eta \\
    \text{s.t.} \quad & f(x) \leq \eta, \\
                       & g_i(x) \leq 0, \quad i = 1, \ldots, m, \\
                       & x \in X.
\end{align*}
这个新问题在满足原问题的约束条件下最小化 $\eta$,其中 $\eta$ 是 $f(x)$ 的一个上界。因此,这个新问题与原问题是等价的。

\textbf{外逼近算法的算法流程}

通过以下步骤迭代地求解转化后的凸优化问题:

\textbf{步骤一:初始化。}

选择一个初始点 $x^0 \in X$,构造初始外逼近集合 $S^0 = \{x \in X : g_i(x^0) + \nabla g_i(x^0)^T (x - x^0) \leq 0, \forall i\}$。

\textbf{步骤二:求解近似问题。}

在第 $k$ 次迭代中,求解近似问题:
\begin{align*}
    \min \quad & \eta \\
    \text{s.t.} \quad & f(x^k) + \nabla f(x^k)^T (x - x^k) \leq \eta, \\
                       & x \in S^k,
\end{align*}
得到解 $(x^{k+1}, \eta^{k+1})$。

\textbf{步骤三:更新外逼近集合。}

更新外逼近集合 $S^{k+1} = S^k \cap \{x \in X : g_i(x^{k+1}) + \nabla g_i(x^{k+1})^T (x - x^{k+1}) \leq 0, \forall i\}$。

\textbf{步骤四:终止条件。}

如果 $(x^{k+1}, \eta^{k+1})$ 满足终止条件(如 $\eta^{k+1} - \eta^k < \epsilon$),则停止迭代,输出 $(x^{k+1}, \eta^{k+1})$ 作为最优解;否则,令 $k = k + 1$,返回步骤二。

\begin{algorithm}[!htbp]
    \small
    \caption{外逼近算法}\label{alg:outer_approximation}
    \begin{algorithmic}[1]
        \Procedure{外逼近}{$f, g, X$}\Comment{输入:目标函数 $f$,约束函数 $g$,可行域 $X$}
        \State 初始化 $x^0 \in X$,$S^0 = \{x \in X : g_i(x^0) + \nabla g_i(x^0)^T (x - x^0) \leq 0, \forall i\}$
        \State $k \gets 0$
        \Repeat
            \State 求解近似问题 $\min\{\eta : f(x) \leq \eta, x \in S^k\}$,得到解 $(x^{k+1}, \eta^{k+1})$
            \State 更新 $S^{k+1} = S^k \cap \{x \in X : g_i(x^{k+1}) + \nabla g_i(x^{k+1})^T (x - x^{k+1}) \leq 0, \forall i\}$
            \State $k \gets k + 1$
        \Until{满足终止条件}
        \State \textbf{return} 解 $(x^{k}, \eta^{k})$\Comment{输出:凸优化问题的最优解}
        \EndProcedure
    \end{algorithmic}
\end{algorithm}

当问题转化为凸整数规划时,外逼近方法将与分支定界方法相结合,并结合这个问题的具体形式,具体的组合方式将在第2章与第3章中进行详细地介绍。

\section{本文主要工作}
本文主要工作集中在武器目标分配问题的研究和求解方法上。首先,本文针对经典的武器目标分配问题进行了深入分析,提出了一种线性化建模方法,将非线性问题转化为线性整数规划问题,为后续的求解算法设计奠定了基础。在此基础上,本文设计了一个基于列生成框架的求解算法,并采用外逼近方法求解列生成子问题,有效地解决了指数多列的挑战。

进一步,本文将研究范围拓展到包含复杂资源约束和动态变化的武器目标分配问题,提出了相应的数学模型和评价方式。特别地,针对动态武器目标分配问题,本文采用了非均匀时间离散的方法,建立了一个能够精确求解的模型,并将列生成求解框架进行了相应的推广,以适应动态变化的作战环境。

在算法设计方面,本文针对列生成子问题的特殊性,提出了两种不同的建模方式,并采用外逼近求解框架进行求解。为了提高求解效率,本文还设计了基于问题特性和实际含义的有效不等式,并对经典的列生成方法进行了改进,允许在求解过程中选择非最优子问题和多个改进列,从而加速了算法的收敛速度。

最后,本文通过数值实验验证了所提出的模型和算法的有效性和效率。实验结果表明,本文的研究不仅在理论上对武器目标分配问题的求解方法做出了贡献,而且在实际应用中具有重要的指导意义。
